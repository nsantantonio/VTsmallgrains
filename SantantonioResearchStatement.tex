\documentclass[11pt]{article}
\usepackage{amsmath}
\usepackage{amssymb}
\usepackage[backend=bibtex, style=authoryear]{biblatex}

% \addbibresource{~/Dropbox/TAGreview/PhDreferences.bib}

\textwidth=470pt
\oddsidemargin=0pt
\topmargin=0pt
\headheight=0pt
\textheight=680pt
\headsep=0pt

\title{Statement of Research}


\author{Nicholas Santantonio}
\date{\today}


\newcommand{\gxe}{G$\times$E}
\newcommand{\gxg}{G$\times$G}


\begin{document}

\section*{\centering Statement of Research}
\begin{center} Nicholas Santantonio \end{center}

% \subsection*{Introduction}

% gxg cows
% cover crop to sequester carbon, 
% htp, 
% weather data
% crop modeling,
% abiotic biotic adaptation
% climate change
% biofuel
% sustainable cropping systems. 


% digital ag
% connect with engineering dept. 
% Fed capacity Hatch funds used only for breeding  

% Lead by Professors Royse Murphy and Don Viands, the Cornell forage breeding program has a rich legacy of providing forage crop varieties to farmers in the Northeast US, as well as collaboration nation-wide, a legacy that I hope to contribute to. 

% Forage research has been drastically under-funded compared to food crops, but renewed interest in sustainable farming has brought them back into the limelight. 

\noindent The chile pepper is perhaps New Mexico's most important cultural icon, prompting the foundation of the official state question, ``Red or Green?''. Important in this identity is the pepper variety (e.g. Big Jim, Joe E. Parker), where the peppers are grown (e.g. Hatch for green, Chimayo for dried red), how hot they are, and what kind of heat they have. The New Mexico pod type balances a broad, medium heat with a unique flavor so familiar to native New Mexicans, they can sniff out a chile roast a mile away. 

Increasingly, chile cultivation has expanded into western NM, Luna County, as well as into Colorado, and California, with the aromas of fresh roasted Hatch green chile reaching grocery store parking lots at least as far away as Ithaca, NY (personal observation, 2014-2017). Processing peppers have also become more important as Americans seek a little more spice in their culinary lives. 

The reduction in available manual labor, and the increased cost of that labor threaten chile production in NM unless a viable mechanization method can be deployed for harvesting all pod types for fresh and processing uses. New varieties with adapted plant architectures that enable mechanical harvest  will need bred to keep this industry viable. Proximal sensing has the potential to help solve this problem by monitoring growth and fruit development, selection can be made on branching and fruit set patterning that facilitate mechanical harvesting. 

Stability in production is also important in the chile pepper industry. Variability in production due to disease, water stress, heat and other factors makes the crop risky for farmers to plant when maize, cotton and onion provide more stable annual crop alternatives. Variability also affects consumers, who desire predictability in flavor, flesh and especially heat of the peppers they buy. 

I intend to develop the foundational capabilities to update the chile breeding program at NMSU to utilize $21^\text{st}$ century breeding technology. This will include the collection, storage and accessibility of genome-wide information, pedigrees, high throughput phenotypes collected through proximal sensing, and phenotypes from field trials. 


\subsection*{Grant 1 Short-term gains: Stability of capsaicin and carotenoid production}

Anyone who has been consuming peppers throughout their life knows how variable they can be, within a single cultivar, year, field, or even on single plant. This is puzzling, given the genetics dont vary with an inbred variety, so ostensibly, only the environment changes. However this does not hold true for all peppers. While jalape\~{n}os and green chiles can excite (and disappoint) when it comes to heat, other peppers such as Thai and haba\~{n}ero peppers seem more consistently spicy. This apparent \gxe\, suggests that there may be genetic factors controlling the variability in heat. More importantly, we may be able to predict and select for these dispersion traits. The production of varieties with stable characteristics is of high priority for farmers, as this allows them to more easily secure a market to sell to.

Double hierarchical generalized linear models can allow for modeling genetic effects in the variability, or dispersion, of a trait (or traits). Similarly, hierarchical Bayesian models can be constructed to capture similar genetic heterogeneity in residuals. Using a diverse panel of jalepe\~{n}o and an NM chile cultivars as examples, with a haba\~{n}ero as a control, genetic heterogeneity for dispersion parameters of capsaicin and carotenoid production will be assessed in a irrigated and managed drought stress field trials. Genomic predictability of trait stability will be determined using genome-wide prediction models, and a scan for important loci effecting variability (i.e. vQTL) will be performed. If found to be heritable, I intend to use genomic selection to improve and release a jalepe\~{n}o and a NM chile with improved heat stability at medium heat levels for the respective types. Cayenne types may also be used to investigate and select for stability of carotenoid production in a similar fashion. 

% These types of methodologies are not limited to capsaicin production. 

% Using double heirarc

\subsection*{Grant 2 Long-term performance: Proximal sensing and machine learning for mechanical harvesting}

In order to understand the impact of plant growth and development on quantitative traits, we must be able to monitor plant development to fit genotype specific growth curves through time. Proximal sensing with relatively inexpensive unmanned aerial vehicles and ground robots can collect high-throughput phenotypes related to growth, and can be used construct three dimensional plant architecture models. Such information can be used to augment phenotypic and genotypic information on expensive phenotypes, such as capsaicin or carotenoid content, and predict genetic values when measurement is too costly or otherwise infeasible.

While much progress has been made for dried red chiles, including through mechanical harvesting of green chile is still a challenge due to the necessity to harvest before the crop has dried. While machinery has been developed through collaboration with engineers at NMSU to harvest green chiles, canopy development and fruit set can be optimized to maximize fruit removal while minimizing damage. Modification of canopy architecture can also change quantitative resistance to pathogens such as \emph{Phytophthora capsici} through avoidance, narrow flowering time for uniform maturity, and increase resilience to heat and drought stress.

Machine learning algorithms, such as convolutional neural networks, can be trained to extract useful predictors of phenotypic traits from proximal sensing data, which can in turn be used in quantitative genetics models to predict line performance for multiple traits. Collection of dense phenotypic and genotypic information, along with environmental information, will allow all data experiments generated in a program (and beyond) to be linked, allowing researchers to share information and build predictive models manifold traits, from flavor profiles to mechanical harvesting.

% While collaboration with engineers at NMSU to further improve mechanical harvesting equipment will certainly be necessary.


% processing peppers, such as Cayanne



% Carotenoids are important parts of our diet as provitamin A, and generally increase the aesthetic value of our dishes. 

% Build predictive models for carotenoid production, disease resistance, 



% Flavor profiles.

% 3D plant architecture

% remote sensing to build 3d models. 


\subsection*{Grant 3: Implementation of a 21st century breeding program using pepper as a model organism}

To accelerate improvement of traits associated with response to stress, plant architecture and quantitative disease resistance, the breeding program must take advantage of the latest breeding technologies. While there has been much discussion on genomics assisted breeding efforts, very little literature focuses on the actual implementation. Because pepper is a annual self-pollinated diploid, it makes a relatively simple system for evaluation of new breeding technologies. As a proof of concept, I intend to use pepper as an model organism to ask important questions about how a traditional breeding program can be adapted to a 21st century breeding program. 

A low cost, moderate density genotyping platform will need to be developed to enable genotyping of all lines that are evaluated in the field. Aerial and ground proximal sensing phenotypes will also be collected regularly and used to monitor development, fit growth models and predict plant architecture. A public \emph{Capsicum} database will need to be hosted to store phenotypic trial data, pedigrees, genotypes, proximal sensing images, that can be easily accessed and queried. This will be based on the CassavaBase database architecture, constructed and made publicly available by the Sol Genomics Network at the Boyce Thompson Institute. 

To build a training set for genomic prediction, all materials represented in recent field trials will be genotyped, as well as all new material that enters into the phenotypic pipeline. Field experimental design will then be optimized to leverage all phenotypic information available, using genotypic information to link otherwise unrelated trials. Eventually, rapid-cycle genomic selection using mathematical optimization will be implemented for expedited improvement of quantitative traits.

\subsection*{Research Philosophy}

In the era of data, the sheer amount of testable hypotheses is seemingly limitless. A shift away from small designed experiments to large observational studies on the breeding program (or whole organism) scale is inevitable. A traditional breeding program generates a plethora of phenotypic data that is used to make yearly breeding decisions, and subsequently discarded. If these materials are genotyped, a feasible proposition given the drastic reduction in costs and availability of third party services, these become treasure troves of data for asking questions, as well as making breeding decisions. This does not mean that we should cease the design and execution of experiments to address specific hypotheses, but we cannot ignore the valuable resource of observational data being collected, generally at great expense.

I believe in the collaborative model, where breeding programs do not operate in isolation. They share germplasm, resources, expertise, and most importantly ideas. Ideas, unlike germplasm, also have the merit of being species flexible. 

Ideas presented here are not limited to peppers, indeed they apply to all crop species. I intend to build a collaborative effort at NMSU to aid all breeding programs to build foundational capabilities to increase efficiency of varietal development. New Mexico also provides an important environment for plant breeding that is likely to be experienced by more farms throughout the globe as climate change progresses. Heat, drought, intense storms and hard frosts will be the new norm, and we must work together to defend our food security through accelerated genetic improvement. 

\end{document}
