\pagenumbering{gobble}
\documentclass[11pt, letterpaper]{moderncv}
\usepackage{times, amsmath}

\moderncvstyle{classic}
\moderncvcolor{green}
\usepackage[utf8]{inputenc}
\usepackage[scale=0.75, margin = 0.75in]{geometry}
\geometry{top  = 20mm}

\name{Nicholas}{Santantonio}
\address{240 Emerson Hall, Ithaca, NY 14853}{(505)~412-2738}
\begin{document}
\recipient{Department of Crop and Soil Environmental Sciences}{College of Agricultural and Life Sciences\\Virginia Polytechnic Institute\\ Blacksburg, VA 24061}
\date{\today}
\opening{Dear Hiring Committee,}
\closing{Respectfully yours,}
\makelettertitle

I am eager to apply to the Small Grains Breeding Assistant Professor position at the Crop and Soil Environmental Sciences Department in the College of Agriculture and Life Sciences at Virginia Polytechnic Institute. I am currently a postdoctoral associate at Cornell University, working with Dr. Kelly Robbins on quantitative genetics solutions for plant breeding. I have combined a strong applied background in small grains and forage breeding programs with theoretical quantitative genetics, allowing me to integrate the newest computational technologies into a successful working breeding program.


My ability to ask meaningful plant breeding and genetics research questions and communicate the findings of my research has been demonstrated through peer-review publications and invited talks at scientific conferences. I have shown an ability to obtain extra-mural funding through a collaborative project to investigate genomic selection strategies in alfalfa. Other collaborations range from projects with colleagues within my institution, to partners in CGIAR programs across the globe, on multiple crop species and breeding goals. 

I also have a strong background in teaching and leadership, currently serving as a co-instructor for an advanced graduate-level course on the evolution of genetic modeling in plant breeding, as well as a TA for introductory courses in biology and plant breeding during my graduate degree. Looking forward, I aim to help prepare future students for data-driven plant breeding by teaching courses with a quantitative emphasis. Computational skills are crucial for leveraging the plethora of data currently being generated in plant science and breeding programs and must be incorporated into plant science and genetics curricula at earlier stages. 


In this data-rich century, a breeding program must adapt to leverage computing capabilities and affordable genotyping technologies. By incorporating genome-wide information and proximal sensing, I intend to shift from the traditional $20^\text{th}$ century breeding program model to a data-driven $21^\text{st}$ century breeding program to improve grain and forage yield, quality, scab resistance and adaptations for double cropping. Implementation of these technologies will not be trivial, and I intend to address several logistical issues by using small grains as model organisms to learn from and demonstrate how to effectively transition to a data-driven breeding program. This transition will provide a valuable resource for public outreach, where farmers and consumers can learn how we are adapting the latest technologies toward improving the small-grains that feed our animals and fill our bellies.


I would like to thank the hiring committee for considering my application for the Assistant Professor of Small Grains Breeding. The Small Grains Breeding Program at Virginia Tech has a rich legacy to which I hope to contribute. 

\vspace{3mm}

Sincerely,\\
\includegraphics[height=2cm]{NSsigWhiteBG}\\
Nicholas Santantonio

\end{document}

