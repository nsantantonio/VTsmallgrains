\documentclass[11pt]{article}
\usepackage{amsmath}
\usepackage{amssymb}
\usepackage[backend=bibtex, style=authoryear]{biblatex}

% \addbibresource{~/Dropbox/TAGreview/PhDreferences.bib}

\textwidth=470pt
\oddsidemargin=0pt
\topmargin=0pt
\headheight=0pt
\textheight=650pt
\headsep=0pt

\title{Statement of Research}


\author{Nicholas Santantonio}
\date{\today}


\newcommand{\gxe}{G$\times$E}
\newcommand{\gxg}{G$\times$G}


\begin{document}

\section*{\centering Statement of Teaching}
\begin{center} Nicholas Santantonio \end{center}
% \subsection*{Introduction}

\noindent Plant breeders have traditionally been generalists, combining genetics with a range of plant sciences to identify farmers' needs and produce products to meet those needs. However, the range of skills required in the field is rapidly increasing. Competitive plant breeders are now expected to be proficient in statistics, programming, bioinformatics and computational biology along with the traditional skills in physiology, pathology, agronomy and genetics. As educators, we must decide how to update our curriculum to best prepare students for careers in the post-genomics era of digital agriculture. 

Both generalists and specialists will be needed for integrating genomics and digital agriculture into modern breeding programs. Generalists may require instruction in team management, whereas specialists may need more rigorous courses in areas previously considered outside plant breeding, such as engineering and computer science. Specialists will be vital for collecting and storing large amounts of data that will be used to model and predict crop growth and development. Generalists will coordinate those activities to lead a team that produces new varieties to meet farmers' needs. %The curriculum has lagged in preparing individuals for these positions. 

% building sensing systems to collect large amounts of data, storing that data in a database, and accessing that data to  This information will then be used to make informed decisions about breeding goals and design optimal mating schemes. 

% As educators, we must actively update our curriculum to prepare students to be competitive in the workplace if Cornell is to continue to be a leader in the plant breeding community. This is especially true today, where plant breeding has seen a rapid and unprecedented paradigm shift in recent years away from tinkering with single genes and toward whole genome engineering. 

Due to the wide breadth of the plant breeding discipline, it may be pertinent to develop multiple paths of instruction. While I believe all future scientists will require some degree of graduate level statistics and computational coursework, a more structured series of quantitative and computational courses would benefit students seeking specialization. Students who chose this path would finish graduate school with the comprehension and ability to effectively use the latest computational techniques in plant breeding, including the use of genome-wide information for decision making.




% At the undergraduate level, plant breeding courses at NMSU are rather limited. Heightened interest in food systems has drawn many young people to seek out information about how crops have been genetically manipulated by humans. An earlier course at the 200 level would be ideal to pique interest and motivate students to pursue plant breeding as a profession. However, there is relatively little infrastructure to foster such an interest once it is seeded. A stronger undergraduate curriculum in new and traditional plant breeding technologies and quantitative genetics would increase enrollment in the Plant Science and Genetics degree paths and add to the department's value to the university.

% While single- and oligo-gene mendelian triats are abundant, why should students be limited to this type of genetic variation?  

\subsection*{Teaching Philosophy}

It is important for students to be exposed to plant breeding ideas from multiple perspectives, and they must be given the opportunity to demonstrate critical thinking on different levels. While some students may show analytical thinking and synthesis on exams, others may shine in more hands-on projects. Most mathematical and computational learning occurs through doing, not through watching. The lecture is important to present material in a concise structured manner, but concepts are cemented when the student can reconstruct the ideas on their own time.

% As a student, my most memorable courses were those that had a semester-long project that was presented to the class at the end of the course. 

Longer term projects provide students the opportunity to practice and apply concepts in a more autonomous environment, and are invaluable for assessing comprehension and critical thinking. In addition, regularly assigned homework and hands-on labs are crucial for estimation of the pace and overall understanding of the material presented, such that adjustments can be made where necessary. The greenhouse and field facilities on campus provide opportunities to get students out of the classroom to see plant breeding in practice. Public databases provide resources where students can get experience working with real, and often large, datasets. Computer simulations are useful tools to evaluate comprehension, where in order to simulate a system correctly, the student must understand that system well.

% Hands-on labs are also useful for evaluating individual students, as it becomes easier to spot students that may need extra attention if they are struggling in labs as well as on homework assignments. 

% The greenhouse and field aspects of plant breeding are difficult to present in lecture courses. One of the benefits of having greenhouse and field space on campus is that students can be brought out of the classroom to see what breeding looks like in practice. The computational aspects of plant breeding can be incorporated into the curriculum through homework assignments, labs and semester long projects. 

% Public databases provide invaluable resources where students can find data sets for use in projects that teach them how to work with real, and often large data sets. 


All courses I instruct would contain a term project of relevant complexity to augment exams, in-class labs and homework assignments. For quantitatively oriented courses, these projects would consist of a hands-on computational component, where students either find a dataset or use their own data to explore the ideas covered in the course. They would then be asked to present their results in written and oral formats that mirror typical scientific communication. Projects may be team oriented to promote collaborative skills and project management. For courses without a significant quantitative aspect, term projects may be formulated as a research proposal.%, where the students would be expected to construct a hypothesis, give background information on merit and propose a way to ask their question while also considering potential pitfalls. %These proposals would be guided throughout the semester to ensure students can 

% Any course related to quantitative genetics should have a computational aspect to it, 

% I also intend to incorporate more perspectives on how domestication, selection and seed systems work in different cultures through guest lectures and case studies. Review of how seed systems function in other cultures will cultivate discussion on the benefits and consequences of these systems relative to the US and western European systems. 

\subsection*{Courses}


I propose to teach the Advanced Plant Breeding course indicated in the position announcement every year to meet the teaching requirement of the position. I would also like to develop an advanced quantitative genetics course that would be offered every other year. 

The Advanced Plant Breeding course would target $21^\text{st}$ century plant breeding concepts, focusing on the use of genome-wide information to drive breeding decisions. Starting with basic probability theory and the single locus model, the course would advance through genome-wide association, genomic prediction and selection, as well as selection theory and breeding program optimization. Concepts introduced in class would be reinforced through weekly computational labs and assignments that require students to write their own software to solve computational plant breeding problems. 


% Instead of a textbook, primary literature would be used to introduce key concepts. Weekly computational labs and a would be used to augment student understanding of course material, and a student term project which would be developed throughout the semester. 


Assignments would be required to be submitted as typed documents in Markdown, \LaTeX\ or similar format, to expose students to more effective modes of mathematical communication outside of Microsoft Office. The term project would consist of groups of 2-4 students finding a genotype-phenotype dataset, and working to develop a genotype to phenotype map, assess genomic predictability and determine an optimal breeding scheme throughout the semester. By the end of the semester, students would be able to demonstrate critical thinking of how to apply quantitative genetics theory to plant breeding problems, with the ability to synthesize when given new plant systems or breeding goals. 

The quantitative genetics course would be based on primary literature review, and offered as a 1 or 2 credit hour course that would meet once a week and cover 1-2 papers per week. Advanced concepts to be covered would including coalescent theory, hierarchical Bayesian models, spatial variation, G$\times$E, multivariate and selection indices, longitudinal models and optimal contribution. Specific computational examples would be developed for key topics and provided to students prior to meeting so that methods can be discussed in detail in class. 

Students would be evaluated using a gain in understanding approach. This approach would require students to submit responses to questions developed to guide understanding of the reading materials prior to class, with the ability to resubmit revised answers after attending. Concepts and questions would be sufficiently complex that prior effort and in class learning can be assessed.

% The class would be taught from a historical perspective, allowing the natural progression of plant breeding over the past century to be a teaching tool for how we arrived at the methods we use today. Instead of a textbook, primary literature would be used to introduce key concepts. By the end of the semester, students will be able to think critically about how breeding technology changes through time have impacted the discipline and the societal view thereof. 



% The Advanced Plant Breeding course would include more focus on $21^\text{st}$ century plant breeding concepts. 



% The Plant Breeding (AGRO462) course may make a good place to introduce more quantitative ideas to prospective plant breeding students. Additionally, I would like to teach a more advanced quantitative genetics course to complement other quantitative courses, such as AGRO610 and AGRO670.


% I propose to teach an advanced graduate course in quantitative genetics every year to meet the teaching requirement of the position. 

% This 3-4 credit hour course would be designed to complement the Advanced Plant Breeding course by providing students with the computational capability necessary to . 

% If I were to instruct AGRO462, the course would shift from 20th century breeding concepts in the beginning toward aspects of 21st Century plant breeding and digital agriculture in the latter half of the course. The course would be taught from a historical perspective, allowing the natural progression of plant breeding over the past century to be a teaching tool for how we arrived at the methods we use today. By the end of the semester, students will be able to think critically about how breeding technology changes through time have impacted the discipline and the societal view thereof. 



% The class would be taught from a historical perspective, allowing the natural progression of plant breeding over the past century to be a teaching tool for how we arrived at the methods we use today. 


%  students will be able to think critically about how breeding technology changes through time have impacted the discipline and the societal view thereof. 



 % in which students would use available computational tools to analyze example datasets with a focus on interpretation of results. 


% course that would include a weekly computational lab and .
% This would include emphasis on the digital agriculture aspect of plant breeding toward the second half of the semester. 

% As for the graduate level course, I would like to offer a quantitative genetics theory or statistical genomics 



%Proposal outlines would be submitted half way through the semester to evaluate project appropriateness, and students would be encouraged to seek input from the instructor throughout the semester. Late semester labs would be used for working on projects, where code readability and reproducibility would be emphasized.% with a focus on interpretation of results.% instead of method theory.

% The course topic is flexible to the needs of the section, however, it is my observation that an introductory plant breeding quantitative/computational course would be useful to help prepare students for further quantitative study. This course might also provide some necessary skills to those who choose not to follow a quantitative path, without the rigorous mathematical detail of the PLBRG7160 quantitative genetics course.






% such as Biometrical Genetics and Plant Breeding (AGRO670) and Advanced Crop Breeding (AGRO610).

% If I were to instruct AGRO462, the course would shift from 20th century breeding concepts in the beginning toward aspects of 21st Century plant breeding and digital agriculture in the latter half of the course. The course would be taught from a historical perspective, allowing the natural progression of plant breeding over the past century to be a teaching tool for how we arrived at the methods we use today. By the end of the semester, students will be able to think critically about how breeding technology changes through time have impacted the discipline and the societal view thereof. 

\subsection*{Curriculum}

Currently, most life science students do not acquire in-depth statistics and programming skills until graduate school, impeding their progress while they learn to grapple with these new languages. Moving forward, I would like to work with faculty in CSES, statistics, engineering and computer science to build a quantitative/computational genetics undergraduate curriculum. Genetics is a vast field; it is imperative that students get exposure to and training in the rapidly changing environment in which they will soon be seeking jobs. While ``single-gene'' genetics is still an important field, more and more the community is focused on the complex ``omics'' network that is the foundation of complex organisms. Students in essentially all sub-fields of genetics will need to be able to deal with large datasets, using tailored algorithms to make inferences, predict the unobserved, and guide decision making. Dealing with big data requires skills in programming, linear algebra, statistics and machine learning. 

\medskip

Genomics and digital agriculture are only just starting to change the landscape of food production. Quantitative skills are one of the specializations imperative in plant science, and Virginia Tech must be at the forefront of preparing the individuals who will usher in this new era.

% along with engineering, computer science, operational research and the traditional disciplines
\end{document}
